\documentclass[a4paper,8pt,epsfig]{article}
\title{Multiparameter persistence computation: review}
\author{Paul Snopov}
\date{June, 2022}

\usepackage{amsmath,amssymb,amsthm,amscd, mathtools}
\usepackage{titling}
\setlength{\droptitle}{-15em}

%%% Работа с картинками
\usepackage{graphicx}  % Для вставки рисунков
\graphicspath{{./images/}}  % папки с картинками
\setlength\fboxsep{3pt} % Отступ рамки \fbox{} от рисунка
\setlength\fboxrule{1pt} % Толщина линий рамки \fbox{}
\usepackage{wrapfig} % Обтекание рисунков и таблиц текстом
\usepackage[export]{adjustbox}
\usepackage{caption}
\captionsetup[figure]{labelsep=space}
\usepackage{subcaption}
\usepackage{float}
\usepackage{tikz-cd}
\usepackage{quiver}

\newtheorem{theorem}{Theorem}
\newtheorem{lemma}{Lemma}
\newtheorem{corollary}{Corollary}

\usepackage[bottom]{footmisc}


\begin{document}
	\maketitle
	 In topological data analysis, one may use persistent homology to extract the topological features of the data. Its construction isn't too hard: one needs to build a simplicial complex $K$ over the data, obtain a filtration of the complex (see fig. \ref{fig:filtration}), that is a sequence of its subspaces $K_0 \subseteq K_1 \subseteq ... \subseteq K_n = K$, and then apply the $i$-th homology functor with coefficients in some field $k$ to it, obtaining the next sequence of modules:
	 \[
	 	H_i(K_0) \to H_i(K_1) \to ... \to H_i(K).
	 \]
	 
	 \begin{figure}
	 	\centering
	 	\includegraphics[width=0.8\linewidth]{filtration.png}
	 	\caption{Example of a filtration of a simplicial complex built over the data. Image credit: \cite{image1}}
	 	\label{fig:filtration}
	 \end{figure}
	 
	 In such setting, the algorithm for computing the persistent homology relies on Gaussian elimination and has its roots in the structure theorem for finitely generated modules over a PID. This is because the persistent module, as above, can be seen as a finitely generated module over $k[t]$. This gives the decomposition of persistent homology into the direct sum of indecomposable summands, which are usually called {\it interval modules}. 
	 
	 \begin{figure}
	 	\centering
	 	\includegraphics[width=0.7\linewidth]{multifiltration.png}
	 	\caption{An example of a bifiltration of a complex at coordinate $(3,2)$. Image credit: \cite{image2}}
	 	\label{fig:multifiltration}
	 \end{figure}
	  
	 But in practice we often encounter richer structures that are described by multiple parameters. Such structures may be modeled with  {\it multifiltrations}, like on the fig. \ref{fig:multifiltration}. Such setting is much more complicated \cite{theormulpers}. But a complete solution to the problem of computing multiparameter persistence\footnote{The name "multiparameter persistence" is due to H. Schenck \cite{schenck}, in the work of Carlsson et al. this is called "multidimensional persistence"} exists and is provided by the Gröbner bases.
	 %Сноску можно сделать, что Карлссон называет мультипараметер персистенс как мультидименсионал, мультипараметер же используют ребята из группы Х. Шенка
	
	 
	 Let's give the precise definitions. We say that a topological space $X$ is {\it multifiltered} if we're given a family of subspaces $\{X_v\}_{v \in \mathbb{N}^n}$ with inclusions $X_u \subseteq X_w$ whenever $u \leq w$ so that the diagrams 
	 
	 % https://q.uiver.app/?q=WzAsNCxbMCwwLCJYX3YiXSxbMiwwLCJYX3t2MX0iXSxbMCwyLCJYX3t2Mn0iXSxbMiwyLCJYX3ciXSxbMCwxXSxbMCwyXSxbMiwzXSxbMSwzXV0=
	 \[\begin{tikzcd}
	 {X_v} && {X_{v1}} \\
	 \\
	 {X_{v2}} && {X_w}
	 \arrow[from=1-1, to=1-3]
	 \arrow[from=1-1, to=3-1]
	 \arrow[from=3-1, to=3-3]
	 \arrow[from=1-3, to=3-3]
	 \end{tikzcd}\]
	 
	 commute. We will consider such multifiltered complexes, where each has has a unique minimal critical grade at which it enters the complex. Such multifiltrations are called {\it one-critical} and mostly arise in practice.
	 
	 \iffalse
	 Given a simplicial complex $K$, we may define {\it chain groups} $C_i$ as the free abelian groups on oriented $i$-simplices. We may provide the {\it boundary operator} $ \delta_i C_i \to C_{i-1}$, that connects the chain groups into a {\it chain complex} $C_\bullet$:
	 \[
	 	... \to C_{i+1} \to C_i \to C_{i-1} \to ... .
	 \]
	 And for any chain complex, we may define the {\it $i$th homology group} $H_i$:
	 \[
	 	H_i(C_\bullet) \coloneqq \frac{\mathrm{ker }\; \delta_i}{\mathrm{im }\; \delta_{i+1}}.
	 \]
	 \fi
	 
	 Given a multifiltration $\{ X_u \}_u$, $i: X_u \to X_v$ induces a map $i_*: H_i(X_u) \to H_i(X_v)$ at the homology level. The {\it $i$th persistent homology} $H_i^{pers}$  is the image of $i_*$ for all pairs $i \leq v$.
	 
	 As was mentioned above, in the setting with a single filtration, persistent homology corresponds to a graded $k[t]$-module. In the same way, persistent homology in the multifiltered setting corresponds to a finitely generated $n$-graded module over $k[t_1, ..., t_n]$. Moreover, the next theorem holds:
	 
	 \begin{theorem}[Realization \cite{theormulpers}]
	 	Let $k = \mathbb{F}_p$, $i \in \mathbb{N}$, $M$ be an $n$-graded module over $k[t_1, ..., t_n]$. Then there's a multifiltered finite simplicial complex $X$ so that $H_i^{pers}(X, k) \cong M$.
	 \end{theorem}
 
 	 One can build the {\it $i$th chain module} over $k[t_1, ..., t_n]$ of a multifiltered complex $\{K_u\}_u$ as
 	 \[
 	 	C_i = \bigoplus\limits_u C_i(K_u),
 	 \]
 	 where the $k$-module structure is provided via the universal property of direct sum and $x^{v-u}: C_i(K_u) \to C_i(K_v)$ is the inclusion $K_u \to K_v$. For one-critical filtrations, these modules are free; the {\it standard basis} for the $i$th chain module $C_i$ is given by the set of $i$-simplices in critical grades.
 	 
 	 Then, given standard bases, we may write the boundary operator $\delta_i: C_i \to C_{i-1}$ explicitly as a matrix with polynomial entries. This gives us a new $n$-graded chain complex that encodes the multifiltration. The homology of this chain complex is precisely the persistent homology of the multifiltration. By definition, homology can be computed in three steps:
 	 \begin{enumerate}
 	 	\item Compute $\mathrm{im}\; \delta_{i+1}$: this problem is the {\it submodule membership problem}, which may be solved by computing the {\it reduced Gröbner bases} using the $\texttt{Buchberger}$, reduction and division algorithms.
 	 	\item Compute $\mathrm{ker}\; \delta_i$: the {\it (first) syzygy module} can be computer using \texttt{Schreyer's} algorithm
 	 	\item Compute $H_i$: one the above two tasks are complete, this is simple: we need to test whether the generators of the syzygy submodule are in the boundary submodule.
 	 \end{enumerate}
 	 
 	 The submodule membership problem is a generalization of the {\it Polynomial Ideal Membership Problem}, which is \texttt{Exspace}-complete. But the multifiltrations provide the additonal structure that is used to simplify the algorithms; they key property is {\it homogeneity}: a matrix $M$ with monomial entries is {\it homogeneous} if:
 	 \begin{enumerate}
 	 	\item every column $f$ of $M$ is associated with a coordinate in the multifiltration $u_f$, and thus a corresponding monomial $x^{u_f}$,
 	 	\item every non-zero element $M_{jk}$ may be expressed as the quotient of the monomials associated with column $k$ and row $j$. resp.
 	 \end{enumerate}
  	Any vector $f$ endowed with a coordinate $u_f$ that may be written as above is {\it homogeneous}.
  	
  	With this in mind, one can simplify the algorithms \cite{compmulpers}:
  	
 	\begin{lemma}
 		For a one-critical multifiltration, the matrix of $\delta_i: C_i \to C_{i-1}$ written in terms of the standard bases is homogeneous.
 	\end{lemma}
 	\begin{corollary}
 		For a one-critical multifiltration, the boundary matrix $\delta_i$ in terms of the standard bases has monomial entries.
 	\end{corollary}
 	\begin{lemma}
 		The S-polynomial $S(f,g)$ of homogeneous vectors $f$ and $g$ is homogeneous.
 	\end{lemma}
 	\begin{lemma}
 		The reminder of the division of homogeneous vector $f$ by the tuple of the homogeneous vectors $(f_1, ..., f_t)$ is homogeneous.
 	\end{lemma}
 	\begin{theorem}[homogeneous Gröbner]
 		The $\texttt{Buchberger}$ algorithm computes a homogeneous Gröbner basis for a homogeneous matrix.
 	\end{theorem}
  	\begin{theorem}[homogeneous syzygy]
  		For a homogeneous matrix, all matrices encountered in the computation of the syzygy module are homogeneous.
  	\end{theorem}
  
  	Using optimization techniques (e.g. proper data structures), one can achieve the following result:
  	\begin{theorem}[\cite{compmulpers}]
  		Multiparameter persistence may be computed in polynomial time.
  	\end{theorem}
	 
	 \begin{thebibliography}{5}
	 	\bibitem{compmulpers} Carlsson, G., Singh, G., Zomorodian, A. (2009). Computing Multidimensional Persistence. In: Dong, Y., Du, DZ., Ibarra, O. (eds) Algorithms and Computation. ISAAC 2009. Lecture Notes in Computer Science, vol 5878. Springer, Berlin, Heidelberg.
	 	
	 	\bibitem{theormulpers} Carlsson, G., Zomorodian, A. The Theory of Multidimensional Persistence. Discrete Comput Geom 42, 71–93 (2009).
	 	
	 	\bibitem{image1} Chi Seng Pun, Kelin Xia, Si Xian Lee Persistent-Homology-based Machine Learning and its Applications -- A Survey. 
	 	
	 	\bibitem{image2} Heather A. Harrington, N. Otter, H. Schenck, U. Tillmann Stratifying Multiparameter Persistent Homology. SIAM J. Appl. Algebra Geom 3, 439-471 (2019)
	 	
	 	\bibitem{schenck}  Harrington, Heather A. and Otter, Nina and Schenck, Hal and Tillmann, Ulrike Stratifying multiparameter persistent homology
	 	
	\end{thebibliography}
	 
\end{document}